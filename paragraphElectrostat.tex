\paragraph{Electrostatic Potential}

The strongest non-bonded forces are the electrostatic interactions
between charged atom pairs. The position of the charges reproduces
the electrostatic potential of the corresponding electron density
distribution. The Coulomb interaction between two charges $q_{i}$
and $q_{j}$ and separated by the distance $r_{ij}$ reads:

\begin{equation}
U_{Coulomb}(r_{ij})=\frac{q_{i}q_{j}}{4\pi\varepsilon_{0}r_{ij}},
\end{equation}
with $\varepsilon_{0}$ the free space permittivity. Two more contributions
are added to the potential: the interaction with an ``artificial''
dielectric medium (reaction field) that represents the induced electric
field created by charges that are located outside the cut-off radius,
and a distance independent term (constant reaction field) that doesn't
contribute to the forces, but serves to reduce the noise from cut-off
effects by eliminating the electrostatic interactions at the cut-off
radius.

\paragraph{Lennard-Jones Potential}

Besides the electrostatic interactions all atoms and molecules (also
the neutral ones) attract and repel each other depending on the distance
between them: when two atoms approach, they start to experience an
attraction \textemdash{} called Van der Waals or VdW attraction \textemdash{}
until their electron clouds begin to overlap, where the quantum mechanical
Pauli repulsion starts to act. In MD the potentials derived from both
interactions are combined in a single, experimentally derived, expression
called the Lennard-Jones (LJ) potential $U_{LJ}$:

\[

\begin{aligned}C_{ij}^{(12)}=\sqrt{C_{ii}^{(12)}C_{jj}^{(12)}} & ,\qquad C_{ij}^{(6)}=\sqrt{C_{ii}^{(6)}C_{jj}^{(6)}}\end{aligned}

\]

\begin{equation}
U_{LJ}(r_{ij})=\frac{C_{ij}^{(12)}}{r_{ij}^{12}}-\frac{C_{ij}^{(6)}}{r_{ij}^{6}},
\end{equation}
where $C_{ij}^{(12)}$ and $C_{ij}^{(6)}$ are parameters that are
determined for each atoms pair $ij$ as a function of the atom and
molecule type according to:

The parameters $C_{ii}^{(12)}$, $C_{jj}^{(12)}$, $C_{ii}^{(6)}$
and $C_{jj}^{(6)}$ are experimentally determined for the different
force fields.

Although other models for the same interactions (VdW and Pauli repulsion)
exist, the LJ potential is commonly used in simulations because it
is computationally cost-effective \textendash{} the first term in
the sum can be determined by a powering the second one.