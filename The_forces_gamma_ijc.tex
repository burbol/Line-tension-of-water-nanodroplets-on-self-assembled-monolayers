The forces $\gamma_{ij}\cdot dl$ acting on a line element $dl$ have
to balance for a stable drop to form. The vertical component of $\gamma_{LG}\cdot dl$
is counterbalanced by a normal and opposite force from the solid acting
against deformation thus leaving only the horizontal components as
stated by Young's equation:

\begin{equation}
\gamma_{SL}+(\gamma_{SG}\cos\theta)-\gamma_{SG}=0\label{eq:horizontal-surf_tens}
\end{equation}
Young's equation \cite{young:1805} yields a condition for the contact
angle $\theta$ shown in Figure \ref{fig:Young_eq}:

\begin{equation}
\cos\theta=\frac{\gamma_{SG}-\gamma_{SL}}{\gamma_{LG}}\label{eq:cos_condition}
\end{equation}
or equivalently:

\begin{equation}
\left|\gamma_{SG}-\gamma_{SL}\right|\leq\gamma_{LG},\label{eq:abs_value_cond}
\end{equation}
from which two cases follow (Figure \ref{fig:c_angle_types}):

\begin{enumerate}
\item $\gamma_{SG}>\gamma_{SL}$ or $\cos\theta>0$, which yields a contact
angle $\theta<90\text{\textdegree}$. In this case the SL-interface
increases at the expense of the SG-interface.
\item $\gamma_{SG}<\gamma_{SL}$ or $\cos\theta<0$, from which $\theta>90\text{\textdegree}$
follows.
\end{enumerate}