\subsection{Discussion}

Our main finding is that the line tension at the three-phase contact
line of a droplet depends on the exact position of the solid-liquid
interface (Figures \ref{fig:tau_phi} and \ref{fig:Tau-vs-MacAngle}).
Therefore, different values are obtained depending on the chosen definition
of interface: The line tension is positive when the SL-interface is
set closer to the solid surface and negative closer to the water bulk.
Gibbs definition of dividing surface yields positions closer to the
water phase (see the density profiles shown in Appendix \ref{chap:apdx-Dens-ptensor}
and the fitted modified Young\textquoteright s equation in Figure
\ref{fig:Fitted-modified-Young} and Appendix \ref{chap:Apdx-Young's-Eq})
and hence negative line tensions. But more values are needed to check
this statement for the whole polarity range. In addition, we found
that the order of magnitude of the line tension is $10^{-12}\mathrm{N}$,
which confirms the precise values found by Guillemot et al. \cite{guillemot:2012}
\textendash{} due to the modified Young's equation \ref{eq: young-eq-mod}
our values also depend on the liquid-gas surface tension $\gamma_{LG}=52.7\mathrm{pN}$
adopted from Ref. \cite{sedlmeier:2008} .

Our results also show that the macroscopic contact angle is independent
of the SL-interface position. Thus, it can be used as a constant for
each surface type (each polarity) to determine the ``best'' definition
of the SL-interface from studies where the results were determined
independently from the SL-interface position (as in \cite{guillemot:2012,vazquez:2008}).
For example, Guillemot et al. \cite{guillemot:2012} measured line
tensions of the order of -30 pN and macroscopic contact angles of
approximately 115\textdegree . Comparing these values with Figure~\ref{fig:Tau-vs-MacAngle}
leads to an SL-interface closer to the liquid bulk. This type of analysis
goes beyond the scope of the present study. However, let us illustrate
the necessity of a new convention for the SL-interface position by
presenting briefly various definitions found in the literature (note
that most of them use positions closer to the solid phase, but this
is not an exhaustive review):

The classical model (Gibbs model \cite{gibbs:1873,gibbs:1948,gumma:2003,levitas:2014,rusanov:2004})
considers the SL-interface position unimportant or uncritical for
the properties at the three-phase contact line, including the line
tension. Thus, it is often ignored \cite{wang:2001,pompe:2002,giovambattista:2007,ruijter:1999}
(although in Ref. \cite{ruijter:1999} a figure shows that the SL-interface
was set closer to the solid surface) or mentioned very briefly as
in Refs. \cite{werder:2003} and \cite{sedlmeier:2008}, where the
authors place the SL-interface simply ``at the surface''. In experimental
studies the SL-interface position is frequently adopted implicitly
from the apparatus used (as in Refs. \cite{lee:2003,extrand:2003})
or from well-established methodologies. For example, in Refs. \cite{amirfazli:2003}
and \cite{gietzelt:2001} they applied a method called ``axisymmetric
drop shape analysis'' or ADSA, which is also used in computer simulations
\cite{ro:1997}. Further analysis is needed to determine how each
apparatus or experimental method defines the SL-interface at the molecular
scale and if it impacts the measurement. But other studies define
it more precisely: Weijs et al. \cite{weijs:2011} apply the definition:
\textquotedblleft$\nicefrac{\sigma}{2}$ above the top row of substrate
atoms\textquotedblright{} where $\sigma$ is ``the characteristic
size of the molecules''; In Refs. \cite{srivastava:2005} and \cite{hautman:1991}
the authors use the same position related to the head groups of the
SAM (the former references the later): ``average height of the CH3
group plus its van der Waals radius'' and ``in the case of the OH-terminated
chains the positions and radii of the 0 atoms'' (the meaning of ``0
atoms'' is unclear); and Sendner et al. \cite{sendner:2009} use
\textquotedblleftthe center of the topmost carbon atoms of the solid
surface''.

Despite the SL-interface definition used, once the (microscopic) contact
angle or the base radius are determined, their value for a different
definition can be calculated using simple trigonometric relations
as shown in Appendix \ref{chap:Analytical-transformation-of}. We
are still working on a simple expression to transform the line tension
directly. But Figure \ref{fig:tau_phi} and \ref{fig:Tau-vs-MacAngle}
already show that this relation is not linear.

Our results lead to various conclusions further: In the first place,
long time intervals are needed for the droplets to equilibrate (see
Appendix \ref{chap:Apdx-theta_t}), ranging from 5ns to more than
100 ns. Secondly, only droplets with high number of molecules yield
spherical shapes. The graphs in Appendix \ref{chap:Apdx-theta_t}
show unstable contact angles for the smallest droplets with 1000 to
2000 molecules, which follow from unstable droplet shapes. The technological
advances have lead to longer simulations and higher number of molecules.
But simulations of less than 1~ns or droplets with less than 300
molecules are still regarded as sufficient and are thus still being
performed \cite{giovambattista:2007,srivastava:2005}. Finally, the
different choices of the SL-interface position may explain in some
cases the difference in sign of the reported line tensions. But additional
factors that can affect not only the sign, but also the order of magnitude.
In their 2004 topic review \cite{amirfazli:2004}, A. Amirfazli and
A.W. Neumann point to ``difficulties in sample preparation, poor
experimental techniques, non-equilibrium, or conceptual and theoretical
difficulties and oversimplifications''. But also that sometimes the
inconsistency may be only apparent due to ``inappropriate comparisons''
between distinct systems.

\subsection{Conclusion}

We have shown that a new convention is necessary for the solid-liquid
interface position in nanoscopic three-phase systems because it impacts
the value and sign of the line tension, together with other microscopic
characteristics like the contact angle or the base radius. We have
also shown that the macroscopic magnitudes, like macroscopic or apparent
contact angle, are independent of that position. It may suffice to
reexamine or redefine the Gibbs dividing surface from a molecular
perspective, but with the actual interpretation it can only be determined
in liquids with extended bulk regions (planar liquid phases), and
not in droplets or in solids. Furthermore, it assumes that the SL-interface
position can be set arbitrarily.