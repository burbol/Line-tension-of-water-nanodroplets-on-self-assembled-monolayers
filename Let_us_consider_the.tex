Let us consider the sign of $\gamma_{ij}$ for different examples
of phases and the interfaces between them:

\begin{itemize}
\item liquid-gas or LG-interface: when $\gamma_{LG}<0$ the liquid molecules
win energy by approaching the interface and entering the gaseous phase.
Thus, $\gamma_{LG}$ must be positive for a stable surface to form.
\item solid-liquid or SL-interface: the sign of $\gamma_{SL}$ depends on
the strength of the interactions. $\gamma_{SL}<0$ if the liquid molecules
experience a stronger attraction from the solid particles than from
the neighboring liquid molecules. But if the liquid interactions are
stronger $\gamma_{SL}>0$.
\item solid-gas or SG-interface: the sign of $\gamma_{SG}$ depends also
on the strength of repulsion/attraction of the gas molecules to the
solid surface. 
\end{itemize}

\subsection{Young's equation and contact angle}

Consider a drop placed on a solid surface and surrounded by vapor.
The three phases (liquid, solid and gas) meet at the ``contact line''
(Figure \ref{fig:Young_eq}). We may think of each interfacial tension
as a force per unit length ($[\gamma_{ij}]=\mathrm{\frac{N}{m}}$)
acting on every point along the contact line tangentially to the surface
(for small drops these forces are much stronger than the gravity,
therefore the latter can be neglected).