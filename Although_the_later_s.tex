Although the later study yields very precise negative values for the
line tension, the motive for the wide difference between studies in
its magnitude remains unclear, as is the theory that would explain
its negative sign (specially in hydrophobic systems).

In the classical theory the position of the interface (the boundary
between two phases) is ambiguously defined. To our knowledge, only
one theoretical study tries to describe the influence of its position
in the line tension\cite{rusanov:2004,rusanov:2005}. Both studies
conclude that it should not influence them, but in their studies made
no distintions regarding its order of magnitude. In experiments and
computer simulations the position is often chosen without regarding
its possible impact on the results. This may become an issue in the
nanoscale because of the sensitivity to small position changes. But
experiments or computer simulations focusing on this question are
still missing.

Figure \ref{fig:Interface-shift} summarizes our hypothesis: lowering
the solid-liquid interface increases the contact angle.

The aim of our study was twofold: first, to determine the line tension
from the contact angle of multiple nanoscopic water drops placed on
hydrophilic and hydrophobic self assembled monolayers; and, secondly,
to test on both magnitudes the impact of shifting the solid-liquid
interface, which might reveal a dependence historically regarded as
irrelevant or nonexistent. To this purpose we chose molecular dynamics
simulations, an ideal tool for the research of nanoscopic systems
(The majority of experiments are limited to the microscopic scale,
while in the theoretical analysis either approximations are made to
distinguish between the micro- and macroscale or data from experiments
is substituted in the derived equations \cite{schimmele:2007,solomentsev:1999}).

The advantage of our approach is its simplicity: we removed interactions
between the atoms of the surface, so that only the water molecules
could interact with them; the solid surfaces were stiff. This eliminated
vibrations, deformations or elasticity effects; the surfaces were
completely smooth (no roughness effects); the hydroxyl head groups
were evenly distributed yielding axial-symmetric drops without uneven
pining effects; the simulations were performed at approximately constant
room temperature. Our results are reliable because we applied well
established force filed parameters included in the GROMACS software
package that have been tested and upgraded over the years \cite{oostenbrink:2004}.
Furthermore, and despite the big computational cost required, we not
only prolonged the simulations up to 100 ns to allow the droplets
to equilibrate completely but also used a wide range of droplets sizes
between 30 $\mathrm{nm^{3}}$ and 300 $\mathrm{nm^{3}}$.