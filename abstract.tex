The line tension acts at the three-phase contact line of droplets
on solid surfaces. It is often determined indirectly through the contact
angle. The line tension is only significant in nanoscopic systems,
but there is not a consensus about its sign and order of magnitude.
The position of the solid-liquid interface is considered non-decisive
in its determination. We investigated whether different positions
would lead to different contact angles and line tensions by performing
MD simulations of differently sized nanodroplets placed on rigid self
assembled monolayers with varying polarities. Analyzing the radial
density profiles, the contact angle of the equilibrated droplets was
determined. A line tension was obtained for each polarity by fitting
the modified Young's equation to the contact angles and base radii.
Three positions of the solid-liquid interface were tested, including
the Gibbs dividing surface. We found that the sign of the line tension
and the microscopic characteristics of the droplet (contact angle,
base radius, etc.) change depending on the interface definition used,
but not the macroscopic contact angle. This may explain in some cases
the inconsistencies of the line tension sign, and also shows the necessity
of a reviewed convention for the solid-liquid interface position at
the molecular scale.

There is not a consensus about the sign and the order of magnitude
of the line tension. It acts at the three-phase contact line of droplets
on solid surfaces, but it is only significant in nanoscopic systems.
Often, it is determined indirectly through the contact angle, for
which the position of the solid-liquid interface is considered non-decisive. 


There is not a consensus about the sign and the order of magnitude
of the line tension. Furthermore, it is difficult to measure experimentally
because it is only significant in nanoscopic systems. For droplets
on solid surfaces it is measured indirectly through the contact angle,
for which the position of the solid-liquid interface is considered
non-decisive.


We investigated whether different positions of the solid-liquid interface
would lead to different contact angles and line tension in nanoscopic
systems. We performed MD simulations of differently sized nanodroplets
on rigid self assembled monolayers with varying polarities. Analyzing
the radial density profiles, the shape and contact angle of the equilibrated
droplets were determined, which were used to calculate the line tension.
Three positions of the solid-liquid interface were tested, including
the Gibbs dividing surface. We found that the sign of the line tension
and the microscopic properties of the droplet (contact angle, base
radius, etc.) change depending on the interface definition used, but
not the macroscopic contact angle. This may explain in some cases
the inconsistencies of the line tension sign, and also shows the necessity
of a reviewed convention for the solid-liquid interface position at
the molecular scale.