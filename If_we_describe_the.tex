If we describe the drop as a spherical cap with radius $R_{C}$ and
base radius $r_{b}$ (see Figure \ref{fig:Spherical-cap.}) we can
express the length of its contact line as $L=2\pi r_{b}$, and the
interfacial areas as $A_{LG}=2\pi R_{C}(1-\cos\theta)$ and $A_{SL}=\pi r_{b}^{2}$.
Then, substituting into equation \ref{eq:deriv_free_energy} and minimizing
$G$ with respect to $\theta$ we obtain the Modified Young equation:

\begin{equation}
\cos\theta=\cos\theta_{\infty}-\frac{\tau}{\gamma_{LG}}\frac{1}{r_{b}},\label{eq: young-eq-mod}
\end{equation}
where $\theta_{\infty}$ indicates the macroscopic or apparent contact
angle defined as

\begin{equation}
\cos\theta_{\infty}=\cos\theta(r_{b}\rightarrow\infty).\label{eq:def_apparent_c_angle}
\end{equation}
This angle $\theta_{\infty}$ is often introduced in experimental
setups where it is difficult or impossible to zoom in to the contact
line itself.

In the following, we will suppress the word ``microscopic'' and
simply use the term ``contact angle'' (instead of ``microscopic
contact angle'') for the angle $\theta$ defined in equations \ref{eq:horizontal-surf_tens}
and \ref{eq:cos_condition}. If we refer to the macroscopic contact
angle $\theta_{\infty}$ we will explicitly indicate it.

\subsection{Gibbs dividing surface}

On the macroscopic scale we can easily imagine the SL-interface between
a water drop and a solid surface (like a ...) as a surface infinitely
thin (with zero thickness) and place its position. But on the molecular
level this becomes a complicated task. We should give then some thickness
to the interface, at least the diameter of the molecules that form
the interface or, even maybe a few molecular layers. For its position
different approaches are taken. But all of them are based on Gibbs
definition, the Gibbs dividing surface (GDS). the position, for example,
of the SL-interface between a

common definition of the interface position is the Gibbs dividing
surface (GDS) The definition of the Gibbs dividing surface (GDS) is
commonly used to determine the position of the interfaces.

We have no reasonable means to determine where the liquid ends or
the solid starts we require an assumption or a convention but we dont
know how thick the interface is (beaker of water??)

A common definition of the interface position is the Gibbs dividing
surface (GDS).

Consider the interface between a water drop and a solid surface. On
the macroscopic scale each phase seems delimited by a clear line.
But the liquid molecules are moving constantly producing different
conformations. It is through their average density that we can establish
a position for the interface. A common convention used, specially
for liquid phases, is Gibbs mathematical definition of the dividing
interface (GDS) \cite{gibbs:1948}, which gives the interface a zero
thickness and is also defined for the general case of two or more
components like, for example, two different liquids partially mixed.
Gibbs defined the interface between two bulk phases $i$ and $j$
with bulk densities $\rho_{i}^{b}$ and $\rho_{j}^{b}$ as an infinitely
thin surface (with zero thickness) whose position depends on the concentration
of each phase and their adsorption. In our simulations the phases
do not mix. We can hence consider only one phase in the GDS definition
and, using the graph in Figure~\ref{fig:GDS-def}, simplify the definition
of its position $z_{GDS}$ to the condition that both areas must be
equal: $A=B$. It follows then that

\begin{equation}
A=\intop_{-\infty}^{z_{GDS}}\rho(z)dz=\intop_{z_{GDS}}^{\infty}(\rho^{b}-\rho(z))dz=B.
\end{equation}
Then, taking into account that the bulk region has a finite size $L^{bulk}$
the equation can be simplified to

\begin{equation}
z_{GDS}=L^{bulk}-\intop_{0}^{L^{bulk}}\frac{\rho(z)}{\rho^{b}}dz.\label{eq:GDS-final}
\end{equation}
A more detailed description of the GDS can be found in Refs. \cite{gumma:2003,levitas:2014,lang:2012}