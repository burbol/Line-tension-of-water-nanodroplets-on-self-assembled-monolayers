\section{Simulations}

\subsection{Molecular dynamics}

Computer simulations offer us insights to length scales of the order
of $10^{-9}\mathrm{m}$ (the nanometer scale) and time scales of the
nanosecond order. They fill the gap between theory and experiment
because they can test the approximations of the theoretical models
and predict events or conditions to be expected in the experiments.
As M. Ripoll put it \cite{ripoll:2011}: `` {[}simulations{]} are
sometimes referred as theoretical simulations by experimentalists
or as computer experiments by theoreticians''.

The techniques of MD have greatly evolved since their first implementation
in the 1950s and are now regarded as the most powerful tool to model
the motion of particles in equilibrated and non-equilibrated systems.

Berni Alder is considered the ``inventor'' of Molecular Dynamics
(he was awarded for that with the Boltzmann Medal in 2001). Together
with his colleague Tom Wainwright, he simulated in the early 1950s
systems with up to 500 particles on two different computers: the UNIVAC
and the faster IBM-704 -- the UNIVAC was the first one with a storing
memory and the IBM-704 the first mass-produced one \cite{alder:1957,alder:2009}

Their work implied the simultaneous solution of up to 500 equations
(one for each interacting particle, see next section), which can only
be solved numerically with computers. At that time Monte Carlo Simulations
(MC simulations) were already being performed (not only by Alder and
his colleagues), but these offer insights only to already equilibrated
systems. Thus, MD appeared out of necessity and has since become an
indispensable tool.

\subsubsection{Computation of particle trajectories }

A MD simulation starts with an initial configuration where all particle
positions and velocities are known. Then, all the positions (and velocities)
are re-calculated in small time steps using Newton's equation of motion
for $N$ particles:

\begin{equation}
m_{i}\frac{\partial^{2}{r}_{i}}{\partial t^{2}}={F}_{i},\enspace i=1,...,N\label{eq:Newton-Equation}
\end{equation}
where $t$ is the time, ${F}_{i}$ the resulting force on particle
$i$, ${r}_{i}$ its position and $m_{i}$ its mass.

The force fields in MD simulations are conservative, thus ${F}_{i}$
is determined from the potential energy $U=U(r_{1},...,r_{N})$ (with
the help of an integration algorithm) by applying:

\begin{equation}
{F}_{i}=-\frac{\partial U}{\partial\mathbf{r}_{i}},\enspace i=1,...,N,\label{eq:Force-PotEnergy-derivative}
\end{equation}
which leads to $N$ equations to be solved simultaneously at every
time step $\triangle t$. Thus, the position vectors ${r}_{i}$
are written as a unique vector $\mathbf{r}$ -- with $3N$ coordinates
instead of $N$ separate vectors ${r}_{i}$ with 3 coordinates
-- to simplify the simultaneous solution in a single step.

\paragraph{Leap-frog algorithm }

The leap-frog integration algorithm or ``integrator'' is used to
solve differential equations numerically. It is applied in MD to solve
Newton's equations of motion (\ref{eq:Newton-Equation}) and hence
obtain the trajectories of the system particles. To determine the
positions at each time step the following equations are applied recursively
as in Figure \ref{fig:Leapfrogs}:

\begin{equation}
\mathbf{r}(t+\triangle t)=\mathbf{r}(t)+\triangle t\mathbf{v}(t+\frac{1}{2}\triangle t),\label{eq:leapfrog-r}
\end{equation}

\begin{equation}
\mathbf{v}(t+\frac{1}{2}\triangle t)=\mathbf{\mathbf{v}}(t-\frac{1}{2}\triangle t)+\frac{\triangle t}{m}\mathbf{F}(t),\label{eq:leapfrog-v}
\end{equation}
where $\mathbf{r}$ and $\mathbf{v}$ denote the positions and velocities
of the particles, $\mathbf{F}$ the resulting force acting on them,
and $t$ the time. At the initial time $t_{0}$ the velocities from
the previous step $t=t_{0}-\frac{1}{2}\triangle t$ are also needed.
If they are not defined it is common to generate them as a Maxwell-Boltzmann
distribution using random numbers.

\begin{center}
\begin{figure}[H]
\centering{}%
\begin{minipage}[t]{0.55\columnwidth}%
\begin{center}
\includegraphics[width=1\columnwidth]{subfiles/MDSimulations/Images/leapfrog}
\par\end{center}
\protect\caption[Leap-frog algorithm.]{\label{fig:Leapfrogs}Leap-frog algorithm. $t$indicates the time,
$t_{i}$ the $i$th time step with $t_{i+1}=t_{i}+\Delta t$, $r_{i}$
and $v_{i}$ the position and velocity at the step $i$. The steps
or ``leaps'' in the positions are shifted by half a time step $\frac{1}{2}\triangle t$
compared with the velocities. This behavior gives the name to the
algorithm.}

\par

\subsubsection{Kinetic energy, temperature and pressure}

In MD simulations the temperature $T$ is defined as in thermodynamics
with the relation:

\begin{equation}
\frac{1}{2}N_{df}k_{B}T=E_{kin},\label{eq:Temp-KinEnergy-FreedomDeg}
\end{equation}
with $k_{B}$ the Boltzmann's constant, $E_{kin}$ the kinetic energy,
and $N_{df}$ the (after removing the movement of the center of motion $N_{com}$and
the constraints imposed $N_{c}$ , $N_{df}~=~3N-~N_{c}-~N_{com}$). On the other hand, the kinetic energy depends on the particle velocities. Its definition for $N$ particles with velocities $v_{i}$ is:

\begin{equation}
E_{kin}=\frac{1}{2}\sum_{i=0}^{N}m_{i}v_{i}^{2}.\label{eq:kinetic_Energy}
\end{equation}
Thus, the system temperature depends on the particle velocities and
these have to be regulated in order to control the temperature.

The pressure tensor $\mathbf{P}$ is defined for anisotropic systems
using the virial and kinetic tensors $\mathbf{\Xi}$ and $\mathbf{E}_{kin}$\cite{parker:1954}
with:

\begin{equation}
\mathbf{P}=\frac{2}{V}(\mathbf{E}_{kin}-\mathbf{\Xi})\label{eq:pressure-tensor}
\end{equation}
and

\begin{equation}
\mathbf{\Xi}=-\frac{1}{2}\sum_{i<j}\mathbf{r}_{ij}\otimes\mathbf{F}_{ij}.\label{eq:virial-theorem}
\end{equation}

Here $\mathbf{r}_{ij}$ and $\mathbf{F}_{ij}$ are the vectors (tensors)
representing, respectively, the distance and the force between particles
$i$ and $j$, and $\mathbf{r}_{ij}\otimes\mathbf{F}_{ij}$ their
tensor product (also called direct product). For isotropic systems
the scalar pressure $P$ is used: $P=\frac{1}{3}\textrm{trace}(\mathbf{P})$.
Note that the trace of the kinetic tensor is identical to the scalar
kinetic energy from equation \ref{eq:kinetic_Energy}.

\subsection{Numerical algorithms}

We refer the reader to the GROMACS manual \cite{d.-van-der-spoel:2010}
for the exact implementation of the following algorithms, which we
will only describe in general terms. The software package GROMACS
allows the user to choose also other algorithms, but they should deliver
the same or equivalent trajectories (with minor differences) for our
system.

\subsubsection{Energy minimization}

Before starting the computation of the particle trajectories we usually
perform what is called an energy minimization, which consists of finding
a local minimum of the potential energy. This ensures that the system
starts in a mechanically stable conformation without, for example,
atom overlaps. However, if the physics of the system is sound the
minimization should only produce minor changes or no change at all.

The steepest (or gradient) descent algorithm is used for finding function
minima. Since the potential energy function in a standard simulation
depends on many interacting particles or variables, the global minimum
will be impossible to find. But for the practical purpose of the energy
minimization a local minimum (a saddle point where all first derivatives
are zero and the second derivatives are positive) already suffices.

Taking into account that the force is the negative derivative of the
potential energy (equation \ref{eq:Force-PotEnergy-derivative}),
the gradient descent algorithm takes the form:

\begin{equation}
\mathbf{r}(=\mathbf{r}_{i}+\frac{\mathbf{F}_{i}}{\max(\left|\mathbf{F}_{i}\right|)}h_{i},\label{eq:steepest}
\end{equation}
where $\mathbf{F}_{i}$ are the forces acting on the particle positions
$\mathbf{r}_{i}$ at the time step $i$, $\max(\left|\mathbf{F_{i}}\right|)$
the largest force component in absolute value, and $h_{i}$ a constant
factor that varies after every step depending on the proximity of
the minimum: it is bigger if the potential energy is approaching the
minimum and smaller in the opposite case. At every step $i$ the forces
$\mathbf{F_{i}}$ are calculated first and then used to update the
positions$\mathbf{r}_{n}$ with equation \ref{eq:steepest}. This
process is repeated until the desired precision is met.

\subsubsection{Temperature coupling}

MD defines the simulations inside NVE ensembles (with constant particle
number N, volume V, and energy E). However, it is common to calculate
the particles trajectories inside NVT ensembles (constant number N,
volume V, and temperature T) as in our droplet simulations. To maintain
the temperature constant, we used the Berendsen algorithm or ``thermostat''
\cite{berendsen:1984}, which reproduces a weak-coupling to a heat
bath with temperature $T_{0}$ by correcting every temperature deviation
from $T_{0}$ according to:

\begin{equation}
\frac{dT}{dt}=\frac{T_{0}-T}{\tau_{temp}}.\label{eq:Berendsen}
\end{equation}
In this manner, the temperature deviation will decay exponentially
with the time constant $\tau_{temp}$. The particle velocities will
be also rescaled, because they are directly related to the kinetic
energy and the temperature through equations \ref{eq:Temp-KinEnergy-FreedomDeg}
and \ref{eq:kinetic_Energy}.

\subsubsection{Pressure coupling}

MD simulations can also be performed in an NPT ensemble (constant
number N, pressure P, and temperature T) where the system is coupled
to a \textquotedblleft pressure bath\textquotedblright . Since the
pressure is directly related to the particle positions through equation
\ref{eq:virial-theorem}, the algorithms for pressure coupling will
also scale the coordinates and the box vectors. The scaling can be
applied uniformly (isotropic pressure coupling), independently in
the x-y and z dimensions (semi-isotropic pressure coupling), or independently
in all directions (anisotropic pressure coupling).

In our simulations we used both the Berendsen and the Parrinello-Rahman
algorithms or ``barostats'': the former to equilibrate the water
droplets alone before they were positioned on top of the solid surfaces,
and the latter to simulate planar water phases (which were used to
determine the GDS) using a semi-isotropic pressure coupling.

The Berendsen barostat \cite{berendsen:1984} maintains the pressure
at a constant reference value $P_{0}$ according to:

\begin{equation}
\frac{dP}{dt}=\frac{P_{0}-P}{\tau_{p}},
\end{equation}
where $\tau_{p}$ is a time constant. While the Parrinello-Rahman
\cite{nose:1983,parrinello:1981} algorithm applies the following
matrix equation of motion for the box vectors represented by the matrix
$\mathbf{b}$:

\begin{equation}
\frac{d\mathbf{b}}{dt^{2}}=V_{box}\mathbf{C^{-1}b'^{-1}}(\mathbf{P}-\mathbf{P_{0})},
\end{equation}
with $V_{box}$ the volume of the simulation box, and $\mathbf{C}$
a matrix parameter determined by the strength of the coupling. $\mathbf{P}$
and $\mathbf{P_{0}}$ are the current and reference pressure matrix,
respectively.