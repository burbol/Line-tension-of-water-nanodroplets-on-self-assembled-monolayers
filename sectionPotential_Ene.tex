\section{Potential Energy}

Forces are the negative derivatives of the potential energy (as in
equation \ref{eq:Force-PotEnergy-derivative}). Thus, to compute the
particle trajectories the potential energy is expressed as a position
dependent function for every interaction. In general, the potential
energy in an MD simulation is composed of the contributions from bonded
and non-bonded interactions. The bonded term comprises covalent bond
and bond-angle interactions, as well as (improper and torsional) dihedral
angle interactions; the non-bonded, the sum of electrostatic and short-ranged
interactions.

Interaction parameters for every atom and molecule are grouped into
different force fields depending on their characteristics.. This includes
charges, bonds and Lennard-Jones parameters. In our study we used
the SPC/E water model (see Section \ref{sub:SPCE}) and the GROMOS
53A6 force field \cite{oostenbrink:2004,oostenbrink:2005}. The reason
for this choice is that the former reproduces water polarization effects
and the latter is optimized for reproducing the condensed phase properties
of alkanes (as well as the free enthalpies of hydration and solvation
for a range of compounds).

A brief review of the potentials that apply to our systems follows.
For a more extensive description we refer the reader to the GROMACS
manual \cite{spoel:2010} (The GROMACS software package is an extension
of the older GROMOS package \cite{scott:1999} designed for biomolecular
MD simulations).

\paragraph{Periodic boundary conditions}

At the edges of the simulation box artificial or unexpected effects
may take place. The solution is usually to set periodic boundary conditions
(PBC). Figure~\ref{fig:PBC-a} shows a two-dimensional equivalent
periodic system, where the central square represents the simulation
box. When a particle reaches one side of the box, it enters again
from the opposite side. This is specially useful when simulating bulk
systems. The minimum image convention is applied to avoid the interaction
of a particle with its own periodic image (Figure \ref{fig:PBC-a}).
In this way, each particle can only interact with the particles and
periodic images inside a region of the same size and shape of the
simulation box.

\subsubsection{Non-bonded Interactions}

\paragraph{\label{sub:Cut-off-Radius}Cut-off radius}

The interaction range of the non-bonded forces is proportional to
$r_{ij}^{-\alpha}$ (with $\alpha$ a positive integer), therefore
infinitely long, which can produce the waste of computational resources.
This is also the case for the short-ranged forces, although their
strength decreases rapidly after only a few molecular diameters ($\sim$1
nm). A cut-off radius $r_{cut}$ is hence introduced \cite{fangohr:2000,frenkel:2002}:
when the distance between two particles $i$ and $j$ exceeds the
cut-off radius, the force between them is set to $\mathbf{F}_{ij}=0$.
Thus, a spherical volume $\frac{4}{3}\pi r_{cut}^{3}$ around each
particle \textendash ~represented in Figure \ref{fig:PBC-b}~\textendash{}
delimits the number of other particles that can interact with it.
The cut-off radius truncates the potential for separations greater
than half the simulation box, $r_{cut}<\frac{1}{2}L_{box}$ because
of the the minimal image convention. Its length can be optimized for
each system, but a value of approximately 1 nm is usually chosen (as
in our systems), because it is the approximate distance from which
the non-bonded forces start to be negligible and would hence let the
effective force unchanged. Inside the simulation box each particle interacts with $N-1$ particles.
But this number goes to infinity when including all the particle images
from the PBC. Using the minimal image convention results in the computation
of a total of $\frac{1}{2}N(N-1)$ pair interactions at each time
step (which is a high number when $N$ is of the order of a few thousand
atoms or molecules). The cut-off radius reduces the total number of
interactions for each particle by a factor of $\frac{4}{3}\pi\left(\frac{r_{cut}}{L_{box}}\right)^{3}$.