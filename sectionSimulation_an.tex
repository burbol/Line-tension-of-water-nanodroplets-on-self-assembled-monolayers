\section{Simulation analysis}

Our analysis method consists of fitting a sphere cap to the droplet
shape and determining the (microscopic) contact angle geometrically.

\subsection{Complete wetting}

In the systems with $\Phi=66\%$ and $\Phi=50\%$ contact angle is
set as $\theta$ = 0 because complete wetting is observed. Thus, they
will left out of the remaining calculations and graphs.

\subsection{Solid-liquid interface position}

The three different positions used as SL-interface $z_{SL}$ are determined
from two-dimensional density profiles. These are produced by dividing
the simulation box in 1000 slices (parallel to the SAM) and averaging
the molecules in each one. Figure \ref{fig:SL-positions} shows the
density profiles of two droplets with 9000 molecules placed on a hydrophobic
and a hydrophilic SAM (with $\Phi=0\%$ and $\Phi=33\%$). The droplet
profile is rounded and continuous. But the SAM profile is composed
of discrete peaks because its atoms are fixed and unmoving. Close
to the surface the droplets also show a layered structure with a high
peak. Its position $z_{W}$ is used as SL-interface, so as the position
of SAM density peak that is closest to the water: $z_{SAM}$. The
third SL-interface position used is the GDS $z_{GDS}$. It is determined
from the density profiles of the water slabs simulations applying
equation~\ref{eq:GDS-final} (Figure \ref{fig:GDSptensor}). These
systems have planar water phases, thus the density takes the bulk
value over an extended region (as opposed to the droplet profiles)
and the GDS definition can be applied (The three SL-interface positions
are indicated in Figure \ref{fig:SL-positions}).