\paragraph{Shifted potentials}

The truncation of forces introduces a discontinuity at the cut-off
radius (in the potential energy and its derivatives) and therefore
breaks the conservation of energy for the particles crossing the cut-off
radius. As a countermeasure the potential energy and the force are
shifted as in Figure \ref{fig:Shifted-Potential-Energy} transforming
both in continuous functions that go smoothly to zero at the cut-off
radius. The shifting is done by adding a small term to the potential
function:

\

\begin{equation}
U_{shifted}(r_{ij}) = \left\{   \begin{array}{lr}     U(r_{ij})-U(r_{c}) & : r_{ij}\leq r_{c} \label{eq:Shifted-Potential}\\     0 & : r_{ij}>r_{c}   \end{array} \right. \end{equation}
With
this transformation the derivative of the shifted potential (the force)
remains unchanged.