\subsection{Contact angle and base radius}

At the beginning of every simulation the contact angle $\theta(t_{i})$
and the base radius $r_{b}(t_{i})$ \textendash{} e.g. droplet shape
\textendash{} are still equilibrating. Afterward, these magnitudes
stabilize around some fixed values $\theta$ and $r_{b}$ that we
determine using the following method (we describe the method only
for $r_{b}$ because the equilibration intervals are identical for
both $\theta$ and $r_{b}$):

The (instantaneous) contact angle $\theta(t_{i})$ is calculated for
every interval $\left[t_{i},t_{i+1}\right]$ of 0.5 ns and every position
$z_{SL}$ using the trigonometric relations of Figure \ref{fig:Circle-Trigonometry}.
Figure \ref{fig:C_Angle_time} shows the complete time evolution of
the contact angles corresponding to the three SL-interface positions
$z_{w}$, $z_{SAM}$, and $z_{GDS}$.