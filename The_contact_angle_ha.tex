The contact angle has been studied in macro- and microscopic systems
experimentally \cite{meiron:2004,tadmor:2008,extrand:2003}, analytically
\cite{kwok:1999,decker:1999}, and with computer simulations \cite{werder:2003,gietzelt:2001}.

The line tension \textendash{} postulated by Gibbs over a century
ago \cite{gibbs:1873,gibbs:1948} \textendash{} remains to our days
a controversial topic. Values with different orders of magnitude~\footnote{See Ref. \cite{amirfazli:2004} for a review of most of the studies available in 2004.}
ranging from $10^{-5}$ N to $10^{-12}$ N, as well as with both positive
\cite{weijs:2011,koga:2007,amirfazli:2003,lee:2003,werder:2003} and
negative sign \cite{lin:2012,hienola:2007} can be found in the literature~\footnote{Refs. \cite{pompe:2002} and \cite{wang:2001} are experimental studies where the determined line tension sign changed depending on the hydrophobicity or the temperature.}.

\cite{rusanov:1999,taylor:2005,schimmele:2007,weijs:2011} offer theoretical
analysis; while an experimentally determined negative value can be
found at \cite{hienola:2007}).