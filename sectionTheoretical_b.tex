\section{Theoretical background}

\subsection{Surface tension}

The intermolecular forces on a particle in the bulk region inside
a liquid cancel out in average. But the particles on the interface
\textendash{} the surface region that is contact with another phase
\textendash{} experience an unbalanced attraction from neighboring
molecules around a semi-circle (Figure \ref{fig:liquid-interactions}),
which pulls them toward the interior of the liquid. Thus, to increase
the surface area by $\Delta A$ a work $\varDelta W$ has to be performed
to bring particles from the interior to the surface. It is quantified
for two phases $i$ and $j$ by the interfacial or surface tension
$\gamma_{ij}$:

\[

\gamma_{ij}\coloneqq\frac{\varDelta W}{\varDelta A}\mathrm{,}\,\mathrm{with}\,[\gamma_{ij}]=\frac{\mathrm{J}}{\mathrm{m}^{2}}=\mathrm{\frac{N}{m}}.

\]

For a given volume and $\gamma_{ij}>0$ the fluid system will minimize
its surface area giving rise, for example, to the spherical shapes
of water drops or soap bubbles.