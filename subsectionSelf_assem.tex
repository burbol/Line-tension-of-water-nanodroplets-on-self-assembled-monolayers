\subsection{Self assembled monolayers}

The discovery of the spontaneous assembly of alkanethiols \textendash{}
alkane chains bonded to a thiol (-SH) group \textendash{} on noble
metals in the 1980's gave rise to simple ways of creating surfaces
with practically any desired chemical property. These surfaces, called
self-assembled monolayers (SAMs), are shaped as crystalline-like monolayers.
They are composed of three parts: the end group that attaches to substrate,
the backbone (typically made up of methylene groups, (CH2)n), and
a head group that is often a functional group that determines the
wetting \cite{abbott:1992} and interfacial properties \cite{godin:2004}.

The process of self-assembly plays a role in many biological processes
like protein folding or the formation of cell membranes. It is activated
by a fast adsorption of the head groups into the substrate due to
the affinity between their atoms (bonded interactions), and followed
by a by a phase of slow orientational ordering. Different degrees
of molecular tilt result depending primarily on the strength of the
VdW interactions between the (methylene) carbons of the adjacent chains
\cite{tao:1994}. Finally, a single monolayer with chains of at least
10 carbons covers the surface of the substrate (Figure \ref{fig:SAMstructure}).