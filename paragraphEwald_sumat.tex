\paragraph{Ewald sumation}

Despite its advantages, the shifted potential can modify the Coulomb
potential considerably. These modifications are difficult to determine
and to correct. Thus, the Ewald summation method can be used to add
the \textquotedblleft missing\textquotedblright{} contribution from
the long-range potential (beyond the cut-off radius). In this method,
the electrostatic potential is split into two parts: a short range
contribution from particle-particle interactions and a long-range
contribution that is solved on a grid using the computational algorithm
FFT (Fast Fourier Transform).

\subsubsection{Bonded interactions}

Our systems are composed of liquid water (with constantly moving molecules)
and rigid surfaces with non-moving atoms. Independently of the type
of bonds in the surfaces (which are covalent bonds) their contribution
to the potential energy must vanish, because it depends on the instantaneous
change in position of the atoms caused by the vibration or torsion
of the bonds. Since our atoms are fixed, we can neglect this term
for the surface atoms.

Our water model (the SPC/E model explained in the next section) yields
a similar result for the water molecules: although they are free to
move, the length and the angle of the bonds are fixed. This is accomplished
by the use of restraints in the equation of motion, which reduces
computational time and still delivers good approximate values of the
characteristic traits of liquid water (like density or radial distribution).

\subsection{\label{sub:SPCE}SPC/E water model}

The structure of liquid water is complicated to model, because the
hydrogen bonds in each molecule are influenced by the surrounding
ones yielding complex geometries, vibrations and molecular orbitals.

There are different water models developed for computer simulations.
Each of them focuses on a specific aspect of the real system \cite{berendsen:1987}.
Each one has it strengths and there is no perfect model. The extended
single point charge (SPC/E) model is particularly suited for our study
because it adds a correction term in the energy from the induced polarization
of the water molecules. This improves the interaction between different
phases. It was developed as part of the GROMOS force field to improve
various water characteristics of the older SPC model \cite{berendsen:1981}:
the density, the radial distribution and the diffusion constant. Both
models use the same geometry for the water molecule: a three point
charge (one on each atom), one LJ-interaction site inside the oxygen
atom, an O-H bond length of 1nm and a H-O-H bond angle of 109,45\textdegree .

In its creation the SPC/E model was fitted for liquid water (using
constraints instead of hydrogen bonds) delivering good results for
the water density, radial distribution, and ... {[}check!{]} {[}ref{]}.
Since then (in 1981??--> check year!) it has been tested (used) in
countless simulations.

SPC/E description: The SPC/E model was developed as part of the GROMOS
force field. It improved the SPC model in the density, the radial
distribution and the diffusion constant.

Potential Energy: The potential energy affecting each water molecule
is composed by three terms: Coulomb interactions between partial charges,
oxygen-oxygen LJ interactions and the polarization correction.