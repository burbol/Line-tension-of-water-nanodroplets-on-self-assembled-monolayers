In 2012 Guillemot et al. \cite{guillemot:2012} measured negative
line tensions of -20 pN to -30 pN with precisions higher than $10^{-2}$,
which is probably the highest resolution achieved with the current
experimental methods. They studied water on hydrophobic porous surfaces
(different types of silica) with cylindrical pores of approximately
1-2 nm. By increasing the pressure water penetrated these cavities
forming menisci in the inside. A cycle was closed by lowering the
pressure again and emptying the nanopores through evaporation. These
intrusion-exclusion cycles were rapidly repeated using a self-developed
device \cite{guillemot:2012a}. From the intrusion pressure they
determined the contact angles of the menisci. By introducing a line
tension they were able to describe the drying process correctly
and to determine the value of the line tension from the extruding
pressure. Thus, their results are independent of the position of
the solid-liquid interface (the plane that separates both phases).

The classical theory (Gibbs model \cite{gibbs:1873,gibbs:1948}) defines
the interface position ambiguously as ``arbitrary'' \cite{rowlinson:2002,lang:2012}.
Thus, when drops in three phase systems (like the ones in Figure \ref{fig:C-angle-menisci})
are analyzed in experiments and computer simulations
the solid-liquid interface is set discretionally. The freedom in the
choice is small, only a few angstroms, but in nanoscopic systems it
may already affect the contact angle or the line tension as shown
in Figure \ref{fig:Interface-shift}. To our knowledge, only one theoretical
study (Ref. \cite{rusanov:2004}) tries to describe the influence
of its position on the line tension. In addition, more studies are
needed to confirm the line tensions measured by Guillemot et al.
On the other hand, the cause of the wide difference between studies
remains unclear, as is the theory that would explain and predict the
line tension sign.