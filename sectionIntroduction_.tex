\section{Introduction}

The tension at the surface of a liquid is one manifestation of the
forces that act between molecules. It is the cause that some insects
can rest on the surface of a lake without drowning (Figure \ref{fig:insect}).
The surface tension also acts when both phases (liquid and gas) meet
and are at equilibrium with a third one (solid). In addition, at the
three-phase contact line (Figure \ref{fig:drop_with_line}) the line
tension also plays a role, although it only becomes relevant on the
nanoscale. Surface and line tensions can only be determined indirectly.
Thus, the contact angle is often measured instead. Figure \ref{fig:C-angle-menisci}
shows the contact angle in two systems with three phases at equilibrium:
a meniscus in a capillary tube and a drop on a solid surface.